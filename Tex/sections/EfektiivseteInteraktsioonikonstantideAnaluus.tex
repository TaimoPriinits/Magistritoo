\documentclass[class=article, crop=false]{standalone}
\usepackage{standalone}
\usepackage{mathptmx}
\usepackage{parskip}
\usepackage{graphicx}
\usepackage[estonian .notilde]{babel}
\usepackage{amssymb}
\usepackage[makeroom]{cancel}
\usepackage{amsmath}
\usepackage{braket}
\usepackage{setspace}
\onehalfspacing
\RequirePackage[utf8]{inputenc}
\RequirePackage[T1]{fontenc}
\RequirePackage[hidelinks]{hyperref}
\sloppy
\relpenalty=10000
\binoppenalty=10000
\usepackage{geometry}
\geometry{
	a4paper,
	total={160mm,235mm},
	left=25mm,
	top=20mm,
}
\usepackage{rotating}
\usepackage{systeme}
\usepackage{mathtools}
\usepackage{tikz}
\usepackage{pgfplots}
\newcommand{\mathcolorbox}[2]{\colorbox{#1}{$\displaystyle #2$}}
\usepackage{xcolor}
\usepackage{soul}
\usepackage{caption}
\usepackage{subcaption}

\begin{document}
\section{Efektiivsete interaktsioonikonstantide analüüs}
Efektiivsete interaktsioonikonstantide sõltuvus interaktsioonikonstantidest
\begin{equation}\label{key}
	\begin{split}
		\Phi = & \Phi ( \mu_{11}, \mu_{22}, \mu_{12}) \\
		\Theta_{11} = & \Theta_{11} ( \mu_{11}, \mu_{22}, \mu_{12}, \nu_{11}) \\
		\Theta_{22} = & \Theta_{22} ( \mu_{11}, \mu_{22}, \mu_{12}, \nu_{22}) \\
		\Theta = & \Theta ( \mu_{11}, \mu_{22}, \mu_{12}, \nu_{12}) \\
		\Gamma_{11} = & \Gamma_{11} ( \mu_{11}, \mu_{22}, \mu_{12}, \nu_{11}) \\
		\Gamma_{22} = & \Gamma_{22} ( \mu_{11}, \mu_{22}, \mu_{12}, \nu_{22}) \\
		\Gamma_{12} = & \Gamma_{12} ( \mu_{11}, \mu_{22}, \mu_{12}, \nu_{12})
	\end{split}
\end{equation}
GRAAFIKUD TEISES FAILIS\\
Arvestame interaktsioonikonstantide märgiga $ \mu > 0 $ ja $ \nu < 0 $ ja kirjutame efektiivsete interaktsioonikonstantide positiivsed ja negatiivsed liikmed eraldi välja
\begin{equation}\label{key}
	\begin{split}
		\Phi^{+} = & 1 + \tilde{ \eta} \mu_{11} + \tilde{ \eta} \mu_{22} + \tilde{ \eta}^{2} \mu_{11} \mu_{22} \\
		\Phi^{-} = & - \tilde{ \eta}^{2} \mu_{12}^{2} \\
		\Theta_{11}^{+} = & \tilde{ \eta} \mu_{12}^{2} - \nu_{11} - \tilde{ \eta} \mu_{11} \nu_{11} - \tilde{ \eta} \mu_{22} \nu_{11} - \tilde{ \eta}^{2} \mu_{11} \mu_{22} \nu_{11} \\
		\Theta_{11}^{-} = & - \mu_{11} - \tilde{ \eta} \mu_{11} \mu_{22} + \tilde{ \eta}^{2} \mu_{12}^{2} \nu_{11} \\
		\Theta_{22}^{+} = & \tilde{ \eta} \mu_{12}^{2} - \nu_{22} - \tilde{ \eta} \mu_{11} \nu_{22} - \tilde{ \eta} \mu_{22} \nu_{22} - \tilde{ \eta}^{2} \mu_{11} \mu_{22} \nu_{22} \\
		\Theta_{22}^{-} = & - \mu_{22} - \tilde{ \eta} \mu_{11} \mu_{22}+ \tilde{ \eta}^{2} \mu_{12}^{2} \nu_{22} \\
		\Theta^{+} = & - \nu_{12} - \tilde{ \eta} \mu_{11} \nu_{12} - \tilde{ \eta} \mu_{22} \nu_{12} - \tilde{ \eta}^{2} \mu_{11} \mu_{22} \nu_{12} \\
		\Theta^{-} = & - \mu_{12} + \tilde{ \eta}^{2} \mu_{12}^{2} \nu_{12}
	\end{split}
\end{equation}
Avaldame, mis punktis toimub efektiivsetel interaktsioonikonstantidel märgimuutus
\begin{equation}\label{key}
	\begin{split}
		\Phi^{+} = & \Phi^{-} \\
		1 + \tilde{ \eta} \mu_{11} + \tilde{ \eta} \mu_{22} + \tilde{ \eta}^{2} \mu_{11} \mu_{22} = & - \tilde{ \eta}^{2} \mu_{12}^{2}
	\end{split}
\end{equation}
\begin{equation}\label{key}
	\begin{split}
		\Theta_{11}^{+} = & \Theta_{11}^{-} \\
	\end{split}
\end{equation}
\begin{equation}\label{key}
	\begin{split}
		\Theta_{22}^{+} = & \Theta_{22}^{-} \\
	\end{split}
\end{equation}
\begin{equation}\label{key}
	\begin{split}
		\Theta^{+} = & \Theta^{-} \\
	\end{split}
\end{equation}
\end{document}